\documentclass[a4paper,12pt]{article}
\usepackage[top=2cm, bottom=2cm, left=2cm, right=2cm]{geometry}
\usepackage[utf8]{inputenc}
\usepackage{amsmath, amsfonts, amssymb}
\usepackage{graphicx}
\usepackage{float}
\usepackage{dsfont}
\usepackage[brazil]{babel}
\usepackage{indentfirst}
\DeclareMathOperator{\sen}{sen}
\DeclareMathOperator{\tg}{tg}
\DeclareMathOperator{\cossec}{cossec}
\DeclareMathOperator{\senh}{senh}
\DeclareMathOperator{\tgh}{tgh}
\DeclareMathOperator{\cossech}{cossech}
\newcommand{\limite}{\displaystyle\lim}
\newcommand{\integral}{\displaystyle\int}
\newcommand{\soma}{\displaystyle\sum}
\newcommand{\arr}{\begin{array}}
\newcommand{\farr}{\end{array}}
\newcommand{\eq}{\begin{equation}}
\newcommand{\feq}{\end{equation}}
\newcommand{\eqn}{\begin{eqnarray*}}
\newcommand{\feqn}{\end{eqnarray*}}
\newcommand{\mat}{\begin{bmatrix}}
\newcommand{\fmat}{\end{bmatrix}}
\newcommand{\sys}{\begin{cases}}
\newcommand{\fsys}{\end{cases}}
\title{Diagrama de Cama Elástica}
\author{Geovanni Fernandes Garcia \\ geovanni@usp.br \\ N°USP: 11298560}

%\usepackage{xcolor}
%\pagecolor[rgb]{0.15,0.15,0.15}
%\color[rgb]{1,1,1}

\begin{document}

\maketitle
O espaço-tempo de Schwarzschild em unidades naturais ($c = 1, G = 1$) possui a métrica:

$$ds^2 = -\left(1-\dfrac{2M}{r}\right)dt^2 + \left(1-\dfrac{2M}{r}\right)^{-1} dr^2 + r^2(d\theta^2 + \sen^2\theta d\phi^2)$$

Portanto, o tensor métrico de Schwarzschild é dado por:

$$ g_{\mu\nu } =
\mat -\left(1-\dfrac{2M}{r}\right) & 0 & 0 & 0\\
0 & \left(1-\dfrac{2M}{r}\right)^{-1} & 0 & 0\\
0 & 0 & r^2 & 0\\
0 & 0 & 0 & r^2\sen^2\theta\\
\fmat $$

Primeiramente, pretendemos reduzir a dimensionalidade do nosso problema considerando o tempo $t$ constante e $\theta = \dfrac{\pi}{2}$, assim pode-se eliminar os termos $dt^2$ e $d\theta^2$ da métrica, nos deixando apenas com:

$$ds^2 = \left(1-\dfrac{2M}{r}\right)^{-1} dr^2 + r^2\sen^2\theta d\phi^2$$
$$ds^2 = \left(1-\dfrac{2M}{r}\right)^{-1} dr^2 + r^2\sen^2\left(\dfrac{\pi}{2}\right) d\phi^2$$
$$ds^2 = \left(1-\dfrac{2M}{r}\right)^{-1} dr^2 + r^2 d\phi^2$$

Vamos agora criar um "hiperespaço" através da métrica das coordenadas cilindricas $(R,\phi,z)$:

$$ds_c^2 = dR^2 + R^2d\phi^2 + dz^2,~~z = z(R)$$

Antes de encontrar uma expressão para $z = z(R)$, iremos aplicar uma transformação de coordenadas nesta métrica:

$$ds_c^2 = dR^2 + R^2d\phi^2 + dz^2$$
$$ds_c^2 = \left(\dfrac{dR}{dr}\right)^2dr^2 + R^2d\phi^2 + \left(\dfrac{dz}{dr}\right)^2dr^2$$
$$ds_c^2 = \left[\left(\dfrac{dR}{dr}\right)^2 + \left(\dfrac{dz}{dr}\right)^2\right]dr^2 + R^2 d\phi^2$$

Finalmente, resolveremos para $z = z(R)$ igualando o que foi obtido de ambas as métricas:

$$ds_c^2 = ds^2$$
$$\left[\left(\dfrac{dR}{dr}\right)^2 + \left(\dfrac{dz}{dr}\right)^2\right]dr^2 + R^2 d\phi^2 = \left(1-\dfrac{2M}{r}\right)^{-1} dr^2 + r^2 d\phi^2$$

$$\sys \left(\dfrac{dR}{dr}\right)^2 + \left(\dfrac{dz}{dr}\right)^2 = \left(1-\dfrac{2M}{r}\right)^{-1}\\
R^2 = r^2 \implies R = r
\fsys$$

$$\therefore\left(\dfrac{dR}{dr}\right)^2 + \left(\dfrac{dz}{dr}\right)^2 = \left(1-\dfrac{2M}{r}\right)^{-1} \implies 1 + \left(\dfrac{dz}{dr}\right)^2 = \left(1-\dfrac{2M}{r}\right)^{-1}$$

$$\dfrac{dz}{dr} = \left[\left(1-\dfrac{2M}{r}\right)^{-1} - 1\right]^{\frac{1}{2}}$$

$$dz = \left[\left(1-\dfrac{2M}{r}\right)^{-1} - 1\right]^{\frac{1}{2}}dr$$

$$z(r) = 2\left[\left(1-\dfrac{2M}{r}\right)^{-1} - 1\right]^{\frac{1}{2}}(r-2M)$$

\end{document}